\documentclass{article}
\usepackage{amsmath,amssymb}
\usepackage[unlinked]{daniel}

\title{Problem set 8}
\author{Daniel Arone}
\date{\today}

\begin{document}

\maketitle

\setcounter{section}{8}

\begin{exerciseseries}
	\exercise Let $(C^i,\partial^i)_{i\in I}$ be a family of chain complexes.
	Show that
	\[H_n\left(\bigoplus_{i\in I}C^i\right)\cong\bigoplus_{i\in I}H_n(C^i).\]
	\begin{solution*}
		Notice that by the definitions, the functions $\partial$ are defined by the components, so we have
		\[\im\partial=\bigoplus_{i\in I}\im\partial^i\quad\text{and}\quad\ker\partial=\bigoplus_{i\in I}\ker\partial^i.\]
		Therefore
		\[H_n\left(\bigoplus_{i\in I}C^i\right)=\ker\partial_n/\im\partial_{n+1}=\bigoplus_{i\in I}(\ker\partial_n^i/\im\partial_{n+1}^i)=\bigoplus_{i\in I} H_n^i.\]
	\end{solution*}
	\exercise Let $X$ be a topological space.

	(1) If $f$ is a (not necessarily closed) path in $X$, prove that $f$ is homologous to $-f^{-1}$.

	(2) Let $a,b,c$ be (not necessarily closed) paths in $X$, such that $a*b*c$ is defined, and a closed path.
	Prove that in $H_1(X)$,
	\[[a*b*c]_H=[a]_H+[b]_H+[c]_H.\]
	\begin{solution*}

		(1) I am not sure what homologous means when $f$ and $-f^{-1}$ aren't in $Z_1(X)$.

		We prove however that $f-(-f^{-1})\in B_1(X)$, because I assume that that is what is wanted.
		Let $\sigma_f:\Delta^2\to X$ be the $2$-simplex defined by $\sigma_f(x,y,z)=f(y)$.
		Notice that
		\[\partial\sigma_f=\sigma_f\circ d^0-\sigma_f\circ d^1+\sigma_f\circ d^2=\sigma_f(0,x,y)-\sigma_f(x,0,y)+\sigma_f(x,y,0)\]
		But $\sigma_f(0,x,y)$ is just $f$ and $f(x,y,0)$ is just $f^{-1}$ and $\sigma_f(x,0,y)$ is just the constant $1$-simplex at $f(0)$.
		But every constant $1$-simplex is the image of the constant $2$-simplex at that point, so we can subtract $f(0)$ from it, and still stay in $B_1(x)$.

		(1) We prove that
		\[[a*b*c]_H-[a]_H-[b]_H-c_[H]\in B_1(X).\]
		Let $\sigma_{abc}:\Delta^2\to X$ be the $2$-chain which maps into the degenerate triangle with sides $a$, $a*b*c$, and $b*c$.
		Let $\sigma_{bc}:\Delta^2\to X$ map into the degenerate triangle with sides $b$, $bc$, and $c$.
		Notice that 
		\[\partial\sigma_{abc}=[a]_H-[a*b*c]_H+[b*c]_H,\]
		and
		\[\partial\sigma_{bc}=[b]_H-[b*c]_H+[c]_H.\]
		So $\partial(-\sigma_{abc}-\sigma_{bc})=[a*b*c]_H-[a]_H-[b]_H-[c]_H$, so they are homologous.
	\end{solution*}
	\exercise Show that if $X$ is a deformation retract of $Y$, then $H_n(X)\cong H_n(Y)$ for all $n\ge 0$.
	\begin{solution*}
		Is this not trivial because deformation retracts are homotopies, and homology is homotopy equivalent?
		Is there something which I have misunderstood?
	\end{solution*}
	\exercise Compute the singular homology groups of the topologist's sine curve
	\[X=\{(x,\sin\frac1x)|0<x\ge 1\}\cup\{(0,0)\}.\]
	\begin{solution*}
		Notice that the path-components of $X$ are $X_0=\{(x,\sin\frac1x)|0<x\le1\}$ and $X_1=(0,0)$, so we have 
		\[H_n(X)=H_n(X_0)\oplus H_n(X_1).\]
		Because $X_1$ is a singleton set we have $H_0(X_1)=\bZ$, and $H_n(X_1)=0$ for positive $n$.
		Notice that function $(x,\sin\frac1x)\mapsto x$ is continuous, and has a continuous inverse $x\mapsto(x,\sin\frac1x)$ so $X_0$ is homeomorphic to the half-open interval $(0,1]$.
		But the homology groups of the half-open interval are just $0$, except when $n=0$, when it is $\bZ$.
		So we have $H_0(X)=\bZ^2$, and $H_n(X)=0$ for positive $n$.
	\end{solution*}
	\exercise Show that for the subspace $\bQ\subset\bR$, the relative homology group $H_1(\bR,\bQ)$ is free abelian, and find a basis.
	\begin{solution*}
		The relative homology group being free abelian can be done with the long exact sequence in relative singular homology, but I don't know how to prove that it has a basis $\bQ\setminus\{x_0\}$ for some $x_0\in\bQ$.
	\end{solution*}
\end{exerciseseries}

\end{document}
