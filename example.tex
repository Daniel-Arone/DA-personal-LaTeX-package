\documentclass{article}
\usepackage[utf8]{inputenc} % millä koodauksella tämä teksti tulkitaan (utf8)
\usepackage[T1]{fontenc}    % ääkköset ja muut perusmerkit
\usepackage[finnish]{babel} % suomen kielen tavatus
\usepackage{amsfonts,amsthm,amsmath,amssymb,gensymb}
%\usepackage{thmtools}
%\usepackage{parskip}        % bigskip toimii paremmin
%\usepackage{mathtools}
%\usepackage[obeyspaces,spaces,hyphens,T1]{url} % URL:ien ladonta ja tavutus
%\usepackage{hyperref}
%\usepackage{graphicx}
%\usepackage{biblatex}
%\addbibresource{sources.bib}

%Own .sty file stuff
\usepackage[linked]{daniel}
%\usepackage{tikz}
%\usepackage{xcolor}

%\hypersetup{
%    colorlinks=true,
%    linkcolor=red,
%    filecolor=magenta,      
%    urlcolor=blue,
%}

\begin{document}

\section{Introduction}
This is a package by me and for me.
The command
\begin{verbatim}\usepackage{daniel}\end{verbatim}
grants you the package with the terms in English, and with the option
\begin{verbatim}\usepackage[fin]{daniel}\end{verbatim}
you get all the terms in Finnish.
Terms can also be renamed by hand e.g.
\begin{verbatim}\renewcommand{\theoremterm}{Lause}.\end{verbatim}
Everything has preset colors, but they can be changed easily e.g.
\begin{verbatim}\definecolor{TheoremBg}{RGB}{rrr,ggg,bbb},\end{verbatim}
however if you do change the colors know that you are infringing upon my artistic vision.

\section{Macros}
It's got some macros
\begin{verbatim}

%Basic shortcuts
\def\mbb#1{\mathbb{#1}}
\def\mfk#1{\mathfrak{#1}}

\def\bN{\mbb{N}}
\def\bC{\mbb{C}}
\def\bR{\mbb{R}}
\def\bQ{\mbb{Q}}
\def\bZ{\mbb{Z}}

%Macros
\newcommand{\floor}[1]{\left\lfloor#1\right\rfloor}
\newcommand{\ceil}[1]{\left\lceil#1\right\rceil}
\newcommand{\paren}[1]{\left(#1\right)}
\newcommand{\angles}[1]{\left\langle#1\right\rangle}

\end{verbatim}

\section{Theorem* environments}
There's a \texttt{theorem*} environment. For example the code

\begin{verbatim}
\begin{theorem*}[name]{red}{green}
body text
\end{theorem*}
\end{verbatim}

gives you

\begin{theorem*}[name]{red}{green}
body text
\end{theorem*}

All theorem* environments are numbered with the \texttt{theorem} counter.
There are also the built-in theorem* environments.


They are \texttt{theorem}, \texttt{corollary}, \texttt{lemma}, and \texttt{definition}.
Their look can be tweaked by redefining \texttt{theorem*}.

For example:

\begin{verbatim}
\begin{theorem}[theorem name]
theorem text
\end{theorem}

\begin{corollary}[corollary name]
corollary text
\end{corollary}

\begin{lemma}[lemma name]
lemma text
\end{lemma}

\begin{definition}[definition name]
definition text
\end{definition}
\end{verbatim}

\begin{theorem}[theorem name]
theorem text
\end{theorem}

\begin{corollary}[corollary name]
corollary text
\end{corollary}

\begin{lemma}[lemma name]
lemma text
\end{lemma}

\begin{definition}[definition name]
definition text
\end{definition}

\begin{example}[example name]
	example text
\end{example}

\section{Exercises}

All of the following environments in this section take an optional argument for their title.




\begin{exercise}
Prove that this package is worth using.
\end{exercise}

\begin{linkedhint}
Consider how you can have linked hints at the end using the \texttt{linkedhint} environment
\end{linkedhint}
\begin{linkedsolution}
The proof is simple, consider the all the useful environments and assorted macros here.
You would be stupid not to use my package.
Numbered theorems, numerous macros, and a simple \texttt{exercise} environment, what more do you need?
It also has a \texttt{linkedsolution} environment.
\end{linkedsolution}

\begin{hint}
Consider also the fact that you can have unlinked hints and solutions with the \texttt{hint} and \texttt{solution} environments respectively.
\end{hint}

\begin{hint*}
You can even have unnumbered hints and solutions with the \texttt{hint*} and \texttt{solution*} environments respectively.
\end{hint*}

The \texttt{hint} and \texttt{linkedhint} environments both use the \texttt{hint*} environment, so redefining it is sufficient to change all of their appearances.
The statement symmetrically holds for solutions.

\section{Misc}

\begin{remark} Check out this \texttt{remark} environment\end{remark}

The package of course also has a \texttt{proof} environment.

\begin{proof}
This here proves the existence of that environment.
\end{proof}

All of the environments can be nested, but I cannot guarantee it will look good.
For example, the nested remark

\begin{remark}
Nested remark
\begin{remark}
Look mom I'm nested
\end{remark}
\end{remark}

\section{Hints}

\printhints

\section{Solutions}

\printsols

\end{document}

